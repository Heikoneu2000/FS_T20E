\documentclass{beamer}
\usepackage[ngerman]{babel}
\usepackage[utf8]{inputenc}
\usepackage{setspace} 
\usepackage{amsmath}
\usepackage{amsthm}
\usepackage{pgfplots}
\usepackage{siunitx}
\usepackage{eurosym}
\sisetup{locale = DE}
\sisetup{per-mode=fraction}
% Lade Beamer Stile
\usepackage{beamerthemesplit}
\usetheme{Rochester}
\usecolortheme{crane}
\setbeamercolor{bgcolor}{fg=black,bg=green!15!white}
\title{Beispielaufgabe Volumenausdehnung fester Körper}
\subtitle{Volumenausdehnung}
\author{Heiko Schröter}
\date{\today}

\setbeamertemplate{enumerate item}{\alph{enumi})}
\newtheorem{satz}{Satz}

\begin{document}

%\frame{\titlepage}

\frame
{
  \frametitle{Aufgabe}
Ein vom Dach fallender Dachziegel (m=\SI{1}{\kilo\gram}) schlägt mit einer Geschwindigkeit von \SI{20}{\meter\per\second} auf dem Boden auf.\\
\begin{enumerate}
\item Welche kinetische Energie hat er unmittelbar vor dem Aufschlag?
\item Aus welcher Höhe fiel der Dachziegel ($g\approx\SI{10}{\meter\per\square\second}$)?
\end{enumerate}
}
\frame
{
\frametitle{Lösung}
\begin{enumerate}
\item 
\begin{align*}
W_1=W_{kin}=\frac{1}{2}\cdot m\cdot v^{2}=\frac{1}{2}\cdot \SI{1}{\kilo\gram}\cdot \left( \SI{20}{\meter\per\second}\right)^{2}=\SI{200}{\joule}
\end{align*}
\item
\begin{align*}
W_2&=W_{pot};\quad W_1=W_2\quad\Rightarrow\quad W_{kin}=W_{pot}\quad\Rightarrow\\
&\Rightarrow \frac{1}{2}\cdot m\cdot v^{2}=m\cdot g\cdot \Delta h\quad \Rightarrow\\
&\Rightarrow \Delta h=\dfrac{v^{2}}{2g}\approx \dfrac{\left( \SI{20}{\meter\per\second}\right)^{2}}{2\cdot \SI{10}{\meter\per\square\second}}=\SI{20}{\meter}
\end{align*}
\end{enumerate}
}
\end{document}