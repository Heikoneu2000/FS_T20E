%\newcommand*{\SCRIPT}{}
\newcommand*{\BEAMER}{}

\ifdefined\SCRIPT
	\documentclass{article}
	\usepackage[a4paper, left=2.5cm, right=2cm, top=5.5cm, bottom=2.5cm]{geometry}
	\usepackage[noxcolor]{beamerarticle}
	\usepackage[ngerman]{babel}
	\usepackage[utf8]{inputenc}
	\usepackage{amsmath}
	\usepackage{amsthm}
	\usepackage{graphicx}
	\usepackage{pgfplots}
	\usepackage{pgfplotstable}
	\usepackage{tcolorbox}
	\usepackage{siunitx}
	\sisetup{locale = DE,
	per-mode=fraction}
	\usetikzlibrary{arrows,patterns}
	\usepackage{biblatex}
	\usepackage{xltabular}
	\addbibresource{Schallgeschwindigkeit.bib}
	\usepackage{scrlayer-scrpage}
	\pagestyle{scrheadings}
	\clearpairofpagestyles
	\ofoot{Seite \pagemark}
	\newcolumntype{C}[1]{>{\centering\arraybackslash}p{#1}}
	\DeclareNewLayer[%
	  background,% Hintergrundebene
	  width=\textwidth,% Größe und Position des Textbereichs
	  hoffset=2.5cm,
	  voffset=2.5cm,
	  align=tl,
	  contents={%
	  \sffamily
	    \setlength{\fboxsep}{2mm}% Abstand Box Text
	    \setlength{\fboxrule}{.5mm}% Dicke Linie
	    % vertikale Position korrigieren    
	    \vskip-\fboxsep\vskip-\fboxrule\vskip-\baselineskip
	    % horizontale Position korrigieren
	    \hspace{-\fboxsep}\hspace{-\fboxrule}%
	    \begin{tabularx}{\textwidth}{|>{\bfseries\Large}C{7cm}|X|X|}                                                       
	\hline
	\vspace{0em}\includegraphics[scale=0.5]{logo_RHS.pdf} & Name: & Datum: \\
	\hline
	
	\end{tabularx}
	    %\makebox[\layerwidth][l]{%
	    %  \fbox{\rule{0pt}{\layerheight}\rule{\layerwidth}{0pt}}% Rahmen
	    %}%
	  }
	]{frame}
	\AddLayersToPageStyle{@everystyle@}{frame}
\fi

\ifdefined\BEAMER
	\documentclass{beamer}
	\usepackage[ngerman]{babel}
	\usepackage[utf8]{inputenc}
	\usepackage{siunitx}
	\usepackage{graphicx}
	\usepackage{pgfplots}
	\usepackage{pgfplotstable}
	\usepackage{svg}
	\sisetup{locale = DE,
	per-mode=fraction}
	% Lade Beamer Stile
	\usepackage{beamerthemesplit}
	\usepackage{tcolorbox}
	\usetheme{Rochester}
	\usecolortheme{crane}
	\usetikzlibrary{arrows,patterns,positioning}
	\usepackage{varwidth}
	\usepackage{pst-pdf}
\fi

\title{Ermittlung der Wärmekapazität von Wasser}
\subtitle{Wärmelehre}
\author{Heiko Schröter}
\date{\today}
\titlegraphic{\includegraphics[scale=0.6]{logo_RHS.pdf}}
\setbeamertemplate{enumerate item}{\alph{enumi})}
\newtheorem{satz}{Satz}

\begin{document}

\frame{\titlepage}

\frame
{
  \frametitle{Wie viel Energie benötigt man, um $\SI{1}{\kilogram} $ Wasser um $\SI{1}{\celsius} $ zu erwärmen?}

\textbf{Versuchsdurchführung:}\\
Wir füllen in einen Wasserkocher $\SI{1}{\liter}$ Leitungswasser. Im Wasserkocher wird ein elektronischer Thermofühler angebracht. Nun erhitzen wir das Wasser bis zum Siedepunkt. Während dem Erhitzen zeichnen wir die Temperatur-Zeit-Kurve auf.\\
\begin{figure}
\textbf{Versuchsbeobachtung:}\\
\begin{tabular}{|c|p{6.5cm}|}
\hline 
\rule[-1ex]{0pt}{2.5ex} Anfangstemperatur & $\vartheta_1=$ \\ 
\hline 
\rule[-1ex]{0pt}{2.5ex} Endtemperatur & $\vartheta_2=$ \\ 
\hline 
\rule[-1ex]{0pt}{2.5ex} Zeit bei $\SI{20}{\celsius} $ & $t_{\SI{20}{\celsius}}=$ \\ 
\hline 
\rule[-1ex]{0pt}{2.5ex} Zeit bei $\SI{70}{\celsius} $ & $t_{\SI{70}{\celsius}}=$ \\ 
\hline 
\rule[-1ex]{0pt}{2.5ex} Masse in $\si{\kilogram}$ & $m=$ \\ 
\hline 
\rule[-1ex]{0pt}{2.5ex} Leistung in $\si{\watt}$ & $P=$ \\ 
\hline 
\end{tabular}
\end{figure}
}

\frame
{
  \frametitle{Temperaturkurve}
\begin{tikzpicture}
\begin{axis}[
	xmin = 0, xmax = 240,
	ymin = 0, ymax = 100,  
	width = \textwidth,
	height = 0.65\textwidth,
	xtick distance = 60,
	ytick distance = 10,
	grid = both,
	minor tick num = 1,
	major grid style = {lightgray},
	minor grid style = {lightgray!25},
	legend style={at={(0,1)}, anchor=north west},
	title = Temperaturverlauf beim Erhitzen von Wasser,
	xlabel = Zeit in s,
	ylabel = Temperatur in °C,
]

% plot data line code
\addplot[teal, dotted, very thick] table[x = t, y = T] {wasser_4.dat};

% Linear regression
\addplot[red]
	table[
	x = t,
	y = {create col/linear regression={y=T}}
] {wasser_4.dat};

\addplot[red, only marks] 
	plot[error bars/.cd,
	y dir=minus,y fixed relative=1,
	x dir=minus,x fixed relative=1,
	error mark=none,
	error bar style={dashed, thick }]
coordinates
	{(19.59,20) (138.64,70)};
	\coordinate (A) at ({axis cs:19.59,0}|-{rel axis cs:0,0});
    \coordinate (B) at ({axis cs:138.64,0}|-{rel axis cs:0,0});

% Add legend
\addlegendentry{Messreihe}
\addlegendentry{
	$ T =\pgfmathprintnumber{\pgfplotstableregressiona} \cdot t
	\pgfmathprintnumber[print sign]{\pgfplotstableregressionb}$
};

\end{axis}
  \node[pin={above right:$t_1=19,59$}] at (A) {};
  \node[pin={above right:$t_2=138,64$}] at (B) {};
\end{tikzpicture}
}

\frame
{
  \frametitle{Wie viel Energie benötigt man, um $\SI{1}{\kilogram} $ Wasser um $\SI{1}{\celsius} $ zu erwärmen?}
\textbf{Ergebnis:}\\
Um einen Liter Wasser von $\SI{20}{\celsius} $ auf $\SI{70}{\celsius} $ zu erwärmen, benötigt man mit einem $\SI{2000}{\watt} $-Wasserkocher $t_{\SI{20}{\celsius}} - t_{\SI{70}{\celsius}}$ Sekunden.\\
\vspace{0.3cm}
\textbf{Folgerung:}\\
	\begin{satz}[]
		{$ Leistung = \dfrac{Energie}{Zeit}\Rightarrow P\approx\dfrac{Q}{t}$}
	\end{satz}
Wie kann man die notwendige Energie berechnen?
  \begin{align*}
	Q&=P \cdot t\\
	&=\SI{2000}{\watt} \cdot \SI{119,05}{\second}\\
	&\approx \SI{238.1}{\kilo\joule}
  \end{align*}
}

\frame
{
  \frametitle{Wie viel Energie benötigt man, um $\SI{1}{\kilogram} $ Wasser um $\SI{1}{\celsius} $ zu erwärmen?}

Um $\SI{1}{\kilogram} $ Wasser um $\SI{50}{\celsius} $ zu erhitzen benötigt man eine Energie von $\SI{238.1}{\kilo\joule}$. \\
\vspace{0.3cm}
Um $\SI{1}{\kilogram} $ Wasser um $\SI{1}{\celsius} (\SI{1}{\kelvin}) $ zu erwärmen benötigt man eine Energie von $\SI{4.762}{\kilo\joule}$.
	\begin{alertblock}{Beachte}
		Der Wirkungsgrad eines Wasserkochers liegt bei ca. $\SI{88}{\%}$. $\Rightarrow \SI{4.762}{\kilo\joule}\cdot\SI{88}{\%}=\SI{4.19}{\kilo\joule} $
	\end{alertblock}

}

\end{document}